\chapter{Nomad Quickstart Guide}

In this chapter we provide a brief introduction to \nomad meant
to get users experimenting with the code quickly; for more extensive
details about the underlying theory and implementation of \nomad
consult the previous chapters.  After discussing requirements and
installation instructions we present the basic \nomad pattern for
applying linear differential operators and a few examples.

\section{Requirements}

\nomad is a largely self contained library, depending on only the 
excellent \verb|Eigen| library and relatively recent \verb|C++| compilers.
Specifically \nomad requires at least \verb|Eigen 3.2| and a
\verb|C++| compiler implementing the full \verb|C++11| specification.
Once these dependencies are met, \nomad code is readily introduced 
into any existing program. 

\section{Installation}

\nomad is a header-only library, meaning there are no shared objects
to compile and build before compiling an executable itself.  Note that
this also means that any attempt to link shared objects that have each
included \nomad headers will fail.

For example, consider an existing executable that compiles with
%
\begin{verbatim}
> ./compiler -03 -o main main.cpp
\end{verbatim}
%
Assuming that the compiler meets the above requirements, enabling \nomad 
in this executable requires only the addition of the appropriate paths and
a single \verb|C++| file,
%
\begin{verbatim}
> ./compiler -03 -I/path/to/Nomad -I/path/to/Eigen -o main main.cpp
\end{verbatim}
%
Some compilers do not implement C++11 by default, and in
this case you may have to explicitly enable C++11 and the 
updated Standard Library, for example
%
\begin{verbatim}
>./compiler -03 -std=c++11 -stdlib=libc++ -I/path/to/Nomad -I/path/to/Eigen \
> -o main main.cpp
\end{verbatim}

\section{Applying Linear Differential Operations}

With the \nomad libraries available applying linear differential operators
to functions is straightforward.  Let us consider, for example, the
function
%
\begin{equation*}
f \! \left( x, y, z \right) = \cos \! \left( e^{x} + e^{y} \right) / z
\end{equation*}
%
In order to apply a linear differential operator we have to implement
this function as a \textit{functor} and then pass it to the appropriate
\textit{functionals}.

\subsection{Implementing a Functor}

In \nomad functions are implemented by defining a class
that derives from \verb|base_functor|,
%
\begin{verbatim}
template<class T>
class base_functor {
public:
  virtual T operator()(const Eigen::VectorXd& x) const;
  typedef T var_type;
};
\end{verbatim}
%
where the template parameter \verb|T| is an appropriate automatic 
differentiation variable.  

First-order linear differential operators, for example, require the 
variable \verb|var1|, in which case our function can be implemented as
%
\begin{verbatim}
class example_functor: public base_functor<var1> {
  var1 operator()(const Eigen::VectorXd& x) const {
    var1 v1 = x[0];
    var1 v2 = x[1];
    var1 v3 = x[2];
    return cos( exp(v1) + exp(v2) ) / v3;  
  }
};
\end{verbatim}
%
The full list of functions that have been implemented for automatic
differentiation variables can be found in Section \ref{sec:operators_and_functions}.

Specifying a functor for each automatic differentiation variable 
can be burdensome and it is usually easier to define a template class.  
For example, the above would become
%
\begin{verbatim}
template <typename T>
class example_functor: public base_functor<T> {
  T operator()(const Eigen::VectorXd& x) const {
    T v1 = x[0];
    T v2 = x[1];
    T v3 = x[2];
    return cos( exp(v1) + exp(v2) ) / v3;  
  }
};
\end{verbatim}
%
with
%
\begin{verbatim}
example_functor<var1>
example_functor<var2>
example_functor<var3>
\end{verbatim}
%
defining functors for first, second, and third-order automatic differentiation
variables, respectively.

\subsection{Passing Functors to Functionals}

Linear differential operators in \nomad are implemented as functionals that 
take in a functor as an argument.  The gradient of a function, for example,
is implemented as
%
\begin{verbatim}
template <typename F>
void gradient(const F& f,
              const Eigen::VectorXd& x,
              Eigen::VectorXd& g)
\end{verbatim}

Using \nomad then becomes a matter of instantiating the appropriate
functor and passing it into the desired functional.  The gradient is readily
computed as
%
\begin{verbatim}
int n = 100;
Eigen::VectorXd x = Eigen::VectorXd::Ones(n);
  
double f;
Eigen::VectorXd grad(x.size());
  
example_functor<var1> example;
gradient(example, x, f, grad);

std::cout << grad.transpose() << std::endl;
\end{verbatim}
%
Similarly, the Hessian requires a functor using the second-order automatic
differentiation variable, \verb|var2|,
 %
\begin{verbatim}
int n = 100;
Eigen::VectorXd x = Eigen::VectorXd::Ones(n);
  
double f;
Eigen::VectorXd grad(x.size());
Eigen::VectorXd H(x.size(), x.size());
  
example_functor<var1> example;
hessian(example, x, f, grad, H);

std::cout << H << std::endl;
\end{verbatim}
%
The full list of linear differential operations implanted in \nomad are detailed 
in \ref{sec:differential_operations}.

\nomad also defines auxiliary automatic differentiation types 
(Table \ref{tab:typedefs}).  The most useful of these are the \text{debug} 
variants that enable strict input and output validation of each component function.  
If any numerical problems are encountered then an exception identifying the source 
of the problem is thrown, assisting in any debugging efforts.  For example,
%
\begin{verbatim}
int n = 100;
Eigen::VectorXd x = Eigen::VectorXd::Ones(n);
  
double f;
Eigen::VectorXd grad(x.size());
  
example_functor<var1_debug> example;
try {
  gradient(example, x, f, grad);
} catch(nomad_error& e) {
  std::cout << e.what() << std::endl;
}

std::cout << grad.transpose() << std::endl;
\end{verbatim}
%
In practice this validation can be a substantial computational burden,
and these types should be reserved for debugging purposes only.

The second class of auxiliary types provided to users are the \textit{wild}
types.  Because they often cause serious, if subtle, pathologies in many
algorithms, functions that are not everywhere smooth, such as the absolute
value and the cube root, are disabled by default in \nomad.  Users may
enable them, however, by using the wild types.  For example,
%
\begin{verbatim}
int n = 100;
Eigen::VectorXd x = Eigen::VectorXd::Ones(n);
  
double f;
Eigen::VectorXd grad(x.size());
  
example_functor<var1_wild> example;
gradient(example, x, f, grad);

std::cout << grad.transpose() << std::endl;
\end{verbatim}
